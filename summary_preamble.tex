\usepackage[]{pdfpages} % To include pdf files
\usepackage{subfiles} % To split big projects into smaller files for faster compilation
\usepackage{lmodern}
\usepackage{multicol} % To make multiple columns on one page
\usepackage{tcolorbox}
\usepackage{graphicx}
\usepackage[margin=2cm]{geometry}
\usepackage[table,xcdraw,dvipsnames,svgnames]{xcolor} % More colours
\usepackage{amsmath,amsfonts,amsthm,amssymb} % Math & symbols
\usepackage{booktabs}
\usepackage{longtable}

\usepackage[protrusion=true,expansion=true]{microtype}

\newcommand{\vcentered}[1]{
    \begingroup
    \setbox0=\hbox{#1}
    \parbox{\wd0}{\box0}\endgroup
}

%-----------------------------------
% Table of Contents
%-----------------------------------
\usepackage{tocloft}
\setcounter{tocdepth}{3} % Exclude subsubsections
\addtolength{\cftsecnumwidth}{0pt} % Set width between title of section and number of section
\renewcommand{\contentsname}{\vspace{-20pt}}
\usepackage[colorlinks, linkcolor=black, urlcolor=cyan]{hyperref} % For hyperlinks, automatically created for table of contents
\usepackage{xr-hyper}

%-----------------------------------
% Algorithms
%-----------------------------------
\usepackage{listings} % Mostly for actual code
\definecolor{codegreen}{rgb}{0,0.6,0}
\definecolor{codegray}{rgb}{0.5,0.5,0.5}
\definecolor{codepurple}{rgb}{0.58,0,0.82}
\definecolor{codeorange}{RGB}{236, 124, 48}
\definecolor{codeblue}{RGB}{70, 113, 196}
\definecolor{codered}{RGB}{192, 0, 0}
\definecolor{backcolour}{rgb}{0.95,0.95,0.95}

\lstdefinestyle{mystyle}{
    backgroundcolor=\color{backcolour},
    commentstyle=\color{codegreen},
    keywordstyle=\color{magenta},
    numberstyle=\tiny\color{codegray},
    stringstyle=\color{codepurple},
    basicstyle=\ttfamily\footnotesize,
    breakatwhitespace=false,
    breaklines=true,
    captionpos=b,
    keepspaces=true,
    numbers=left,
    numbersep=5pt,
    showspaces=false,
    showstringspaces=false,
    showtabs=false,
    tabsize=2,
    firstnumber=auto,
    frame=topline,
    frameround=tttt,
}

\lstset{style=mystyle,}

\usepackage[
    linesnumbered,
    lined,
    %ruled,
    vlined,
    %boxed,
    commentsnumbered,
    ]{algorithm2e}% Mostly Pseudocode

\SetAlgoNlRelativeSize{-1}



\usepackage{algpseudocode}
\usepackage{algorithmicx}
\newcommand\mycommfont[1]{\small\ttfamily\textcolor{codegreen}{#1}}
\SetCommentSty{mycommfont}

\SetKw{KwFalse}{\color{magenta}false}
\SetKw{KwTrue}{\color{magenta}true}
\SetKw{KwRet}{\color{red}return}
\SetKw{KwContinue}{\color{magenta}continue}
\SetKw{KwBreak}{\color{magenta}break}
\SetKw{KwDownTo}{down to}
\SetKw{KwAnd}{\color{magenta}and}
\SetKw{KwOr}{\color{magenta}or}
\SetKw{KwNot}{\color{magenta}not}
\SetKw{KwIn}{\color{magenta}in}
\SetKw{KwError}{\color{red}error}
\SetKw{KwNil}{\color{magenta}NIL}
\SetKw{KwPrint}{\color{magenta}print}

\SetKw{KwFloat}{\color{teal}float}
\SetKw{KwInt}{\color{teal}int}
\SetKw{KwString}{\color{teal}String}
\SetKw{KwBool}{\color{teal}bool}
\SetKw{KwChar}{\color{teal}char}

\SetKwFor{For}{\color{codegreen}For}{\color{codegreen}do}{\color{codegreen}end}
\SetKwFor{ForEach}{\color{codegreen}ForEach}{\color{codegreen}do}{\color{codegreen}end}
\SetKwFor{While}{\color{codegreen}While}{\color{codegreen}do}{\color{codegreen}end}
\SetKwIF{If}{ElseIf}{Else}{\color{codegreen}If}{\color{codegreen}then}{\color{codegreen}Else If}{\color{codegreen}Else}{\color{codegreen}end}
\SetKwProg{Fn}{\color{codegreen}Function}{:}{}
\SetKwProg{Cl}{\color{codegreen}Class}{:}{}

%-----------------------------------
% Trees, Skip-Lists etc.
%-----------------------------------
\usepackage{tikz}
\usepackage{tkz-graph}
\usetikzlibrary{matrix,
                backgrounds,
                positioning,
                trees,
                shapes.geometric,
                calc,
                arrows,
                chains,
}

%-------------------------------------
% Captions
%-------------------------------------
\usepackage[hang, small,labelfont=bf,up,textfont=it,up,labelformat=empty]{caption}
% \captionsetup[figure]{name=Skip-List}

\usepackage[]{sectsty}
\usepackage{titlesec}
%------------------------
% Footer and Header
%------------------------
\usepackage[]{fancyhdr} % For footers and header
\pagestyle{fancy}
\usepackage[]{lastpage} %For page reference: Page x out of y

\lhead{} % Left Header
\chead{} % Centre Header
\rhead{} % Right Header

\lfoot{\footnotesize \leftmark } % Left Footer
\cfoot{\footnotesize \rightmark} % Centre Footer
\rfoot{\footnotesize Page \thepage\ of \pageref{LastPage}} % Right Footer - Page Reference

\renewcommand{\headrulewidth}{0.0pt} % Sets size of header rule - here its not visible
\renewcommand{\footrulewidth}{0.4pt} % Sets size of footer rule
%----------------------------------------------
% Title
%----------------------------------------------
\def\MainColor{codered} % Sets the main color of the template
\usepackage{titling}
\newcommand{\HorRule}{\color{Black} \rule{\linewidth}{2pt}}
\newcommand{\TitleRule}{\color{\MainColor} \rule{\linewidth}{15pt} \vspace{-90pt}\linebreak \HorRule \linebreak}
\newcommand{\SectionHorRule}{\color{Black} \rule{\linewidth}{1pt}}

\pretitle{\vspace{-80pt} \fontsize{30}{50} \usefont{OT1}{phv}{b}{n} \TitleRule \color{Black}} % Horizontal rule before the title

	\title{Software Engineering Summary} % The title

	\posttitle{\vskip 0.5em} % Whitespace under the title

\preauthor{\begin{flushleft}\large \lineskip 0.5em \fontsize{15}{0} \usefont{T1}{phv}{b}{n} \color{Black}} % Author font configuration

	\author{Moritz Gerhardt } % Name

	\postauthor{% Anything you might want to add
        \\ \href{https://github.com/moricetamol}{\vcentered{\includegraphics[height=15pt]{Github_logo.png}} \large \texttt{moricetamol}}
		\par\end{flushleft}} % Horizontal rule after the title

\date{} % Add date or leave empty to not show any date !!!\today

%-----------------------------
% Section Style
%-----------------------------

\newcommand{\SectionFont}{\LARGE\bfseries}
\titleformat{\section} % Type of heading to format
%[] % Shape
{\color{\MainColor} \titlerule[15pt] \color{White} \titlerule[7pt] \color{Black} \titlerule[2pt] \SectionFont} % Format
{\thesection} % Label
{1em} % Seperator
{\SectionFont} % Before-Code
[{\titlerule[2pt]}] % After-Code

\newcommand{\SubSectionFont}{\large\bfseries}
\titleformat{\subsection} % Type of heading to format
%[] % Shape
{\SubSectionFont} % Format
{\thesubsection} % Label
{1em} % Seperator
{\SubSectionFont} % Before-Code
[{\color{\MainColor}\titlerule[1pt]}] % After-Code

\newcommand{\SubSubSectionFont}{\bfseries}
\titleformat{\subsubsection} % Type of heading to format
%[] % Shape
{\SubSubSectionFont} % Format
{\thesubsubsection} % Label
{1em} % Seperator
{\SubSubSectionFont} % Before-Code
[{\color{\MainColor!50}\titlerule[1pt]}] % After-Code

\renewcommand{\thesection}{\arabic{section}}
\renewcommand{\thesubsection}{\arabic{section}.\arabic{subsection}}
\renewcommand{\thesubsubsection}{\arabic{section}.\arabic{subsection}.\arabic{subsubsection}}

%----------------------------------
% Graphics
%----------------------------------
\usepackage{etoolbox}
\usepackage{tcolorbox}
\tcbset{
    boxrule=1pt,
    fonttitle=\bfseries,
    width=\textwidth,
    before=\vspace{2pt},
    after=\vspace{2pt},
    inlbox/.style={
        boxsep=0pt, left=6pt, right=6pt, top=2pt, bottom=2pt,
        arc=5pt, 
        before upper={\rule[-3pt]{0pt}{10pt}},
        on line,
        before=\space,
        after=\space
    },
    greenbox/.style={
        colback=codegreen!5!white,
        colframe=codegreen!80!white
    },
    magentabox/.style={
        colback=magenta!5!white,
        colframe=magenta!85!black
    },
    orangebox/.style={
        colback=codeorange!5!white,
        colframe=codeorange!85!black
    },
    bluebox/.style={
        colback=codeblue!5!white,
        colframe=codeblue!95!black
    },
    redbox/.style={
        colback=codered!5!white,
        colframe=codered!85!black
    },
    purplebox/.style={
        colback=codepurple!5!white,
        colframe=codepurple!85!black
    },
    definitionbox/.style={
        redbox % Change these to the colors you want your defboxes to be
    },
    mathematicsbox/.style={
        greenbox % Change these to the colors you want your mathboxes to be
    },
    codingbox/.style={
        bluebox % Change these to the colors you want your code boxes to be
    }
}

\newtcolorbox[]{defbox}[1][]{
    definitionbox,
    title=\IfValueTF{#1}{#1}{}
}
\newtcolorbox[]{defbox*}[1][]{
    definitionbox
}
\newtcolorbox[]{smalldefbox}[1][]{
    definitionbox,
    hbox,
    title=\IfValueTF{#1}{#1}{}
    }
\newtcolorbox[]{smalldefbox*}[1][]{
    definitionbox,
    hbox
}
\newtcbox{\inldefbox}[1][]{
    definitionbox,
    inlbox
}

\newtcolorbox[]{mathbox}[1][]{
    mathematicsbox,
    title=\IfValueTF{#1}{#1}{}
}
\newtcolorbox[]{mathbox*}[1][]{
    mathematicsbox
}
\newtcolorbox[]{smallmathbox}[1][]{
    mathematicsbox,
    hbox,
    title=\IfValueTF{#1}{#1}{}
    }
\newtcolorbox[]{smallmathbox*}[1][]{
    mathematicsbox,
    hbox
}
\newtcbox{\inlmathbox}[1][]{
    mathematicsbox,
    inlbox
}

\newtcolorbox[]{codebox}[1][]{
    codingbox,
    title=\IfValueTF{#1}{#1}{}
}
\newtcolorbox[]{codebox*}[1][]{
    codingbox,
}
\newtcolorbox[]{smallcodebox}[1][]{
    codingbox,
    hbox,
    title=\IfValueTF{#1}{#1}{}
    }
\newtcolorbox[]{smallcodebox*}[1][]{
    codingbox,
    hbox
}
\newtcbox{\inlcodebox}[1][]{
    codingbox,
    inlbox,
    fontupper=\ttfamily
}

\newcommand\defc[1]{{\textbf{\color{\MainColor}#1}}}

\usepackage{parskip} % Removes the indentation of the first line of new section
\usepackage[toc,sort=use]{glossaries}