\documentclass[
../../Software_Engineering_Summary.tex,
]
{subfiles}

\externaldocument[ext:]{../../Software_Engineering_Summary.tex}
% Set Graphics Path, so pictures load correctly
\graphicspath{{../../}}

\begin{document}
\section{General}
\subsection{What is Software?}
Software can describe a lot of things, some examples include:

\begin{defbox*}
    \begin{itemize}
        \item Executable programs
        \item Configuration files
        \item System documentation
        \item User documentation
        \item Support environment
        \item etc.
    \end{itemize}
\end{defbox*}

In general Software can be divided into three categories:

\begin{defbox*}
        \begin{itemize}
        \item{Application Software}
        \begin{itemize}
            \item Interacts directly with the end user
            \item General purpose software (To be used in other applications: Word processing, image processing, etc.)
            \item Customized Software (Software specifically for a specific purpose: CAD, IDE, BIM, etc.)
        \end{itemize}
        \item System Level Software
        \begin{itemize}
            \item Does not interact directly with the end user
            \item Responsible for keeping systems running (Operating System, firmware, drivers, etc.)
        \end{itemize}
        \item Software as a Service (SaaS)
        \begin{itemize}
            \item Runs on a server
            \item Indirectly accessed via client (browser, remote shell, etc.)
        \end{itemize}
    \end{itemize}
\end{defbox*}

Furthermore, some characteristics of Software include:

\begin{defbox*}
    \begin{itemize}
        \item Software does not wear out, its environment does
        \begin{itemize}
            \item Software is subject to continous change in hardware, needs to be able to adapt
            \item Software should be able to support new requirements, use cases etc.
        \end{itemize}
        \item Software often lives longer than anticipanted
        \begin{itemize}
            \item Almost impossible to know use cases in advance as it can be in use for years or even decades (Excel used in biology geneology $\rightarrow$ lead to unexpected behaviour)
        \end{itemize}
        \item Software properties are hard to measure
        \begin{itemize}
            \item How does the code relate to software quality?
            \item How do we measure progress?
            \item How do we measure resilience?
        \end{itemize}
    \end{itemize}
\end{defbox*}
\newpage
\subsection{Software Engineering}
Typically a software is designed to solve different needs of different groups involved in the development of the software.

\begin{defbox*}
    \begin{itemize}
        \item Customer / Client
        \begin{itemize}
            \item Often the person / organisation that'll pay for the development
            \item Sets a budget, timeframe, requirements etc.
            \item $\rightarrow$ Requirements analysis
        \end{itemize}
        \item User
        \begin{itemize}
            \item Usually the person / organisation that'll use the software
            \item Defines what the software is used for and subsequently what requirements this sets
            \item $\rightarrow$ Use Case Analysis
        \end{itemize}
        \item Manufacturer
        \begin{itemize}
            \item Usually the person / organisation that'll design and develop the software
            \item Is concerned with how to build the software in a way that satisfies the customers and users
            \item $\rightarrow$ Domain Modelling, Architecture, Quality Assurance, Design Practice, Verification
        \end{itemize}
        \item Maintainer
        \begin{itemize}
            \item Usually the person / organisation that'll maintain the software during its lifetime
            \item Responsible for maintenance of the software and updates to make it usuable for new demands and requirements
            \item $\rightarrow$ Maintenance and Evolution
        \end{itemize}
    \end{itemize}
\end{defbox*}
After all these aspects are consideres the software system is the built with the specific requirements in budget and time.

There are quite a few problems that can happen with Software:

\begin{defbox*}
    \begin{itemize}
        \item Unexpected Errors:
        \begin{itemize}
            \item Few errors are obvious
            \item Most of them are near impossible to test for and detect (Algorithmic error, arithmetic overflow)
            \item Often go undetected for a long time as they're usually the result of very specifc inputs for complex computations
        \end{itemize}
        \item Altough errors can occur, as long as they do no violate the requirements they are not considered errors: 
        \begin{itemize}
            \item INABIAF: It's not a bug, it's a feature
        \end{itemize}
        \item Most errors are caused by missing verification, validation or documentation.
        \begin{itemize}
            \item This usually indicates an insufficient match between requirements and implementation
        \end{itemize}
    \end{itemize}
\end{defbox*}

Errors can also occur as a result of social aspects:

\begin{defbox*}
    \begin{itemize}
        \item Insufficient validation
        \item Inadequete Specification
        \item Constantly changing requirements
        \item Insuffciently trained software engineers
        \item Managment with lacking grasp on software development
        \item Unsuitable methods, languages, tools etc.
    \end{itemize}
\end{defbox*}

\end{document}