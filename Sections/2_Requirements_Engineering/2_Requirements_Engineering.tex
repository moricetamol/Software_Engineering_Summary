\documentclass[
../../Software_Engineering_Summary.tex,
]
{subfiles}

\externaldocument[ext:]{../../Software_Engineering_Summary.tex}
% Set Graphics Path, so pictures load correctly
\graphicspath{{../../}}

\begin{document}
\section{Requirements Engineering}
In the following we are gonna look at requirements engineering using the case study of a car sharing service.
The main roles and functionalities of a car sharing service are:

\begin{greenbox}[Main Roles \& Functionalities]
    \begin{itemize}
        \item Role-Independent
        \begin{itemize}
            \item Authentication
        \end{itemize}
        \item Administrator
        \begin{itemize}
            \item Add / change new cars, rental locations
            \item Biling
        \end{itemize}
        \item User
        \begin{itemize}
            \item Check availability
            \item Request booking
            \item Change booking
        \end{itemize}
        \item Service Staff
        \begin{itemize}
            \item Take out vehicle for service
        \end{itemize}
    \end{itemize}
\end{greenbox}

Requirements Analysis is concerned with building a system of what the product \textit{needs} to fulfill in terms of budget, time and surrounding criteria.

So the objectives are akin to:

\begin{greenbox}[What has to be developed?]
    \begin{itemize}
        \item Need to understand the problems that arise in the requirement elicitation phase
        \item The different kinds of requirements
        \item The requirement engineering workflow
        \item Modelling requirements
        \begin{itemize}
            \item Scenarios \& Use cases
            \item Notations: Textual and Graphical
        \end{itemize}
    \end{itemize}
\end{greenbox}

Although the objectives seem pretty straightforward, requirements analysis can be tricky due to how ambigous language can be. Thorough communication is important to understand fully understand what the client wants.

\begin{greenbox}[What is Requirements Engineering?]
    The process of 
    \begin{itemize}
        \item finding
        \item analysing
        \item documenting
        \item validating
    \end{itemize}
    software requirements.
\end{greenbox}
\subsection{What are Requirements?}

\begin{greenbox}[Definition]
    \begin{itemize}
        \item Requirements as descriptions of the services provided by the system
        \begin{itemize}
            \item Car booking
            \item Service booking
            \item Location tracking
            \item etc.
        \end{itemize}
        \item Requirements as the operational constraints of the system
        \begin{itemize}
            \item Database throughput
            \item System memory
            \item Navigation systems
            \item etc.
        \end{itemize}
    \end{itemize}
\end{greenbox}

These requirements are usually handled in the form of \textbf{System Requirements Specification (SRS)} Documents (Ger: Pflichtenheft) or \textbf{User stories}, structured natural language of use cases, state diagrams etc. stored in the product backlog (ordered list of requirements)

\subsection{Different Types of Requirements}
Overall the requirements can be divided in the following:

\begin{greenbox*}
    \begin{itemize}
        \item User Requirements
        \item System Requirements
        \item Functional Requirements
        \item Non-Functional Requirements
        \item Domain Requirements
    \end{itemize}
\end{greenbox*}

\subsubsection{User Requirements}

\begin{greenbox*}
    State in language or diagrams:
    \begin{itemize}
        \item What services the system should provide
        \item What the operational constraints are
    \end{itemize}
    The descriptions are often high-level and abstract.
\end{greenbox*}

For example "According to german law, a car sharing service must keep track of all bookings"

\subsubsection{System Requirements}

\begin{greenbox*}
    Precise and detailed specification of the systems
    \begin{itemize}
        \item functions
        \item services 
        \item operational constraints
    \end{itemize}
\end{greenbox*}

For example: "After a successful booking the user must be shown an overview of their booking"

"Booking details must be stored for 10 years"

\begin{greenbox}[Characteristics]
    \begin{itemize}
        \item Refinement of user Requirements
        \item Determine system interface (functional)
        \item Recorded as part of the SRS and part of the contract with the client
        \item Authored by software developer or business analyst with the client
    \end{itemize}
\end{greenbox}

\subsubsection{Functional Requirements}

\begin{greenbox*}
    Functionality that is clearly identifiable and localized in the code
    \begin{itemize}
        \item Services providede by the system
        \item System reactions to inputs or events
        \item System behaviour in specific situations like Network distruption
    \end{itemize}
\end{greenbox*}

\subsubsection{Non-Functional Requirements (NFR)}

\begin{greenbox*}
    Constraints of the services or functions
    \begin{itemize}
        \item Service Level Agreement (SLA)
        \item Constraints from development process 
        \item Alignment to standards (e.g. Protocols)
    \end{itemize}
\end{greenbox*}

NFRs often apply to the whole system as they cannot be handled by simply adding a peice of code.

For example: 
"The database must be able to process 1000 queries a second"
"User data must only be accessible to authorized persons"

\begin{greenbox}[Examples of Non-Functional Requirements]
    \begin{itemize}
        \item Product requirements
        \begin{itemize}
            \item Reliability (crashes, use cases)
            \item Efficiency (performance, memory)
            \item Portability (Not confined to one device or service)
        \end{itemize}
        \item Organisational Requirements
        \begin{itemize}
            \item Delivery mode (beta, continous)
            \item Implementation (Programming language, framework)
            \item Standardization (ISO standards or similar)
        \end{itemize}
        \item External requirements
        \begin{itemize}
            \item Interoperability (TUCaN $\Leftrightarrow$ Moodle)
            \item Ethical aspects
            \item Legal aspects (safety, security, privacy)
        \end{itemize}
    \end{itemize}
\end{greenbox}

NFRs may often result in the identification of functional requirements and are often more important to adhere to strictly than individual functional requirements.

A problem with NFRs come from how subjective they are: What is ethical, what is ease of use, what is good performance etc.?

\subsubsection{Domain Requirements}

\begin{greenbox*}
    Are derived from the application domain rather than the needs of the user
    \begin{itemize}
        \item Often expressed in domain specific language $\rightarrow$ Hard to understand for software engineers. 
        \item For example: Software engineers usually do not have profound knowledge of chemistry, however the client might be a chemist and needs software that can be used for very specific applications. 
        \item Often implicitly assumed as obvious to domain experts
        \item Can be functional or non-functional
    \end{itemize}
\end{greenbox*}

\subsection{Feasibility Study}
The objective of the Feasibility Study is to obtain a justified understanding of whether the requirements engineering and system development phases should be \textbf{started}. This is usually base on:

\begin{greenbox*}
    \begin{itemize}
        \item Business requirements 
        \item Outline description of the system
        \item Description of how the system should support the business
    \end{itemize}
\end{greenbox*}

The resulting \textbf{Feasibility Report} then covers

\begin{greenbox*}
    \begin{itemize}
        \item Whether the system contributes to the objective of the organization
        \item Whether the system can be implemented within technical, financial and schedule constraints
        \item Whether the system can be implemented using other systems used by the company
    \end{itemize}
\end{greenbox*}

\subsection{Requirements Elicitation and Analysis}
\subsubsection{Requirements Discovery}
\begin{greenbox}[Systematic Requirement Discovery\newline Viewpoint-Oriented Approach]
    \begin{itemize}
        \item Interactor Viewpoint
        \begin{itemize}
            \item People or systems who interact directly with the system
            \item End Users, Administrators, Service Personal, etc.
            \item \textbf{Direct Stakeholders}
        \end{itemize}
        \item Indirect Viewpoint
        \begin{itemize}
            \item Stakeholders who influence the requirements, but won't use the system directly
            \item CFO, Data protection personal, etc.
            \item \textbf{Indirect Stakeholders}
        \end{itemize}
        \item Domain Viewpoint
        \begin{itemize}
            \item Domain characteristics \& constraints that influence the requirements
            \item Legal, Ethical, etc.
        \end{itemize}
    \end{itemize}
\end{greenbox}

The goal of the requirement elicitation process is to develop more specific viewpoints and use them to discover more specific requirements.

The elicitation can be done in an interview which are usually structured as follows:

\begin{greenbox}[Systematic Requirement Discovery\newline Interviews]
    \begin{itemize}
        \item Closed Interviews:
        \begin{itemize}
            \item Predefined questions
        \end{itemize}
        \item Open Interviews:
        \begin{itemize}
            \item No predefined agenda 
        \end{itemize}
        \item Interviews should only be used as a supplement:
        \begin{itemize}
            \item Interviewee can be biased
            \item Interviewee can assume domain knowledge
        \end{itemize}
    \end{itemize}
\end{greenbox}

Some further elicitation techniques are:

\begin{greenbox}[Systematic Requirement Discovery\newline Other Techniques]
    \begin{itemize}
        \item Scenario Analysis
        \begin{itemize}
            \item Analyses the sequence of interactions with the system
        \end{itemize}
        \item Use Case Analysis
        \begin{itemize}
            \item Analyses the use cases of the system
        \end{itemize}
    \end{itemize}
\end{greenbox}

\subsubsection{Requirements Classification \& Organisation}
For further structured workflow the requirements should be categorized, this can be done using the \textbf{FURPS+} Model:

\begin{greenbox*}
    \begin{itemize}
        \item \textbf{F}unction
        \item \textbf{U}se
        \item \textbf{R}equirements
        \item \textbf{P}riority
        \item \textbf{S}cope
        \item \textbf{+}
        \begin{itemize}
            \item Implementation
            \item interface
            \item Operations
            \item Packaging
            \item Legal
        \end{itemize}
    \end{itemize}
\end{greenbox*}

\subsubsection{Requirements prioritisation \& Negotiation}
Another problem in the elicitation process are conflicts. Different stakeholders might have different requirements. These conflicts need to be resolves through negotiation.

\newpage
\subsubsection{Requirements Documentation}
The produced requirements are then documented and used as a basis for further elicitation and analysis. These documents (SRS) can be formal or informal.

\begin{greenbox}[SRS Target Groups]
    \begin{itemize}
        \item Client, users
        \item Managers: Client and Manufacturer
        \item System Engineers, system testers, system maintainers
        \item Anyone concerned with ordering, using, manufacturing or maintaining
    \end{itemize}
    The level of detail of the SRS depends on the system, development process, whether the product is developed in-house or external etc.
\end{greenbox}

The usual format of an SRS is:

\begin{greenbox}[System Requirement Specification (SRS) Document Format]
    \begin{enumerate}
        \item Introduction
        \begin{enumerate}
            \item Purpose of the SRS
            \item Scope of the product (Also what isn't in the scope)
            \item Glossary
            \item References
            \item Overview
        \end{enumerate}
        \item General Description
        \begin{enumerate}
            \item Product perspective
            \item Product functions
            \item User characteristics
            \item Limitations
            \item Assumptions and dependencies
        \end{enumerate}
        \item Specific Requirements
        \item Appendices, Index, etc.
    \end{enumerate}
\end{greenbox}

\subsubsection{Requirements Validation}

\begin{greenbox}
    [Requirement Validation Checklist]
    \begin{itemize}
        \item Validity
        \begin{itemize}
            \item Do the requirements capture the needed features?
            \item Is additional functionality needed?
        \end{itemize}
        \item Consistency
        \begin{itemize}
            \item Are the requirments conflicting?
        \end{itemize}
        \item Completeness
        \begin{itemize}
            \item Do the requirements cover all the features and constraints?
        \end{itemize}
        \item Realism
        \begin{itemize}
            \item Can the requirements be implemented feasably?
        \end{itemize}
        \item Verifiability
        \begin{itemize}
            \item Is there criteria to check whether the requirements are met?
        \end{itemize}
        \item Traceability
        \begin{itemize}
            \item Is each requirement tracable to the source of the requirement?
        \end{itemize}
    \end{itemize}
\end{greenbox}
\end{document}